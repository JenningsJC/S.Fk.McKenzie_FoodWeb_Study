% Options for packages loaded elsewhere
\PassOptionsToPackage{unicode}{hyperref}
\PassOptionsToPackage{hyphens}{url}
%
\documentclass[
]{article}
\usepackage{lmodern}
\usepackage{amssymb,amsmath}
\usepackage{ifxetex,ifluatex}
\ifnum 0\ifxetex 1\fi\ifluatex 1\fi=0 % if pdftex
  \usepackage[T1]{fontenc}
  \usepackage[utf8]{inputenc}
  \usepackage{textcomp} % provide euro and other symbols
\else % if luatex or xetex
  \usepackage{unicode-math}
  \defaultfontfeatures{Scale=MatchLowercase}
  \defaultfontfeatures[\rmfamily]{Ligatures=TeX,Scale=1}
\fi
% Use upquote if available, for straight quotes in verbatim environments
\IfFileExists{upquote.sty}{\usepackage{upquote}}{}
\IfFileExists{microtype.sty}{% use microtype if available
  \usepackage[]{microtype}
  \UseMicrotypeSet[protrusion]{basicmath} % disable protrusion for tt fonts
}{}
\makeatletter
\@ifundefined{KOMAClassName}{% if non-KOMA class
  \IfFileExists{parskip.sty}{%
    \usepackage{parskip}
  }{% else
    \setlength{\parindent}{0pt}
    \setlength{\parskip}{6pt plus 2pt minus 1pt}}
}{% if KOMA class
  \KOMAoptions{parskip=half}}
\makeatother
\usepackage{xcolor}
\IfFileExists{xurl.sty}{\usepackage{xurl}}{} % add URL line breaks if available
\IfFileExists{bookmark.sty}{\usepackage{bookmark}}{\usepackage{hyperref}}
\hypersetup{
  pdftitle={Food Web Project Test Markdown},
  pdfauthor={Jennings},
  hidelinks,
  pdfcreator={LaTeX via pandoc}}
\urlstyle{same} % disable monospaced font for URLs
\usepackage[margin=1in]{geometry}
\usepackage{graphicx,grffile}
\makeatletter
\def\maxwidth{\ifdim\Gin@nat@width>\linewidth\linewidth\else\Gin@nat@width\fi}
\def\maxheight{\ifdim\Gin@nat@height>\textheight\textheight\else\Gin@nat@height\fi}
\makeatother
% Scale images if necessary, so that they will not overflow the page
% margins by default, and it is still possible to overwrite the defaults
% using explicit options in \includegraphics[width, height, ...]{}
\setkeys{Gin}{width=\maxwidth,height=\maxheight,keepaspectratio}
% Set default figure placement to htbp
\makeatletter
\def\fps@figure{htbp}
\makeatother
\setlength{\emergencystretch}{3em} % prevent overfull lines
\providecommand{\tightlist}{%
  \setlength{\itemsep}{0pt}\setlength{\parskip}{0pt}}
\setcounter{secnumdepth}{-\maxdimen} % remove section numbering

\title{Food Web Project Test Markdown}
\author{Jennings}
\date{10/27/2020}

\begin{document}
\maketitle

\hypertarget{read-in-raw-data}{%
\subsection{Read in raw data}\label{read-in-raw-data}}

\texttt{Raw\_Data\_2019\_July\textless{}-\ read.csv("\textasciitilde{}/S.Fk.McKenzie\_FoodWeb\_Study/DataRaw/SFMR\_Wisseman\_Long\_Aggregate\_Output\_2019\_July.csv")\ Raw\_Data\_2019\_October\textless{}-\ read.csv("\textasciitilde{}/S.Fk.McKenzie\_FoodWeb\_Study/DataRaw/SFMR\_Wisseman\_Long\_Aggregate\_Output\_2019\_October.csv")}
\#\# Adds A Column named ``Season'' to each seasonal data set

\begin{verbatim}
Season <- rep("Summer",length(Raw_Data_2019_July$Date))
Raw_Data_2019_July$Season<- cbind(Season)
Season <- rep("Fall",length(Raw_Data_2019_October$Date))
Raw_Data_2019_October$Season<- cbind(Season)

## Rbind the dataframes together
\end{verbatim}

CombRawDat\_2019\textless-
rbind(Raw\_Data\_2019\_July,Raw\_Data\_2019\_October)

\begin{verbatim}

## Coerce the dates from factor to date in the "Date" column
\end{verbatim}

class(CombRawDat\_2019\(Date) CombRawDat_2019\)Date\textless-
as.Date(CombRawDat\_2019\(Date, format="%Y-%m-%d") class(CombRawDat_2019
\)Date)

\begin{verbatim}
## Removes the columns: Waterbody, Higher.classification,Common.name, Abundance  
\end{verbatim}

CombRawDat\_2019\textless- CombRawDat\_2019{[},-c(1,10,13,14){]}

\begin{verbatim}
## Subset by: 
- Treatment(Treatment = Site)
- Substrate (Benthic, Submerged Wood), 
- Stage
- Insect
\end{verbatim}

Disturbed\_BenthInsect\_Data\_2019\textless- subset(CombRawDat\_2019,
Treatment==``Disturbed'' \& Substrate==``Benthic'' \& Insect==``insect''
\& Stage==``L'') Flooded\_Forest\_BenthInsect\_Data\_2019\textless-
subset(CombRawDat\_2019, Treatment==``Flooded Forest'' \&
Insect==``insect'' \& Stage==``L'') \#\#Only benthic substrate was
sampled Relic\_Channel\_BenthInsect\_Data\_2019\textless-
subset(CombRawDat\_2019, Treatment==``Relic Floodplain Channel'' \&
Substrate==``Benthic'' \& Insect==``insect'' \& Stage==``L'')
Phase3\_BenthInsect\_Data\_2019\textless- subset(CombRawDat\_2019,
Treatment==``Phase 3'' \& Insect==``insect'' \& Stage==``L'') \#\#Only
benthic substrate was sampled in Phase3
Phase4\_BenthInsect\_Data\_2019\textless- subset(CombRawDat\_2019,
Treatment==``Phase 4'' \& Insect==``insect'' \& Stage==``L'') \#\#Only
benthic substrate was sampled in Phase4
Disturbed\_WoodInsect\_Data\_2019\textless- subset(CombRawDat\_2019,
Treatment==``Disturbed'' \& Substrate==``Submerged Wood'' \&
Insect==``insect'' \& Stage==``L'')
Relic\_Channel\_WoodInsect\_Data\_2019\textless-
subset(CombRawDat\_2019, Treatment==``Relic Floodplain Channel'' \&
Substrate==``Submerged Wood'' \& Insect==``insect'' \& Stage==``L'') ```
\includegraphics{Markdown_test_file_files/figure-latex/pressure-1.pdf}

Note that the \texttt{echo\ =\ FALSE} parameter was added to the code
chunk to prevent printing of the R code that generated the plot.

\end{document}
